%% w-bksamp.tex. Current Version: October 6, 2008
%%%%%%%%%%%%%%%%%%%%%%%%%%%%%%%%%%%%%%%%%%%%%%%%%%%%%%%%%%%%%%%%
%
%  Sample file for
%  Wiley Book Style, Design No.: SD 001B, 7x10
%  Wiley Book Style, Design No.: SD 004B, 6x9
%
%
%  Prepared by Amy Hendrickson, TeXnology Inc.
%  http://www.texnology.com
%%%%%%%%%%%%%%%%%%%%%%%%%%%%%%%%%%%%%%%%%%%%%%%%%%%%%%%%%%%%%%%%



%%%%%%%%%%%%%
% 7x10
%\documentclass{wileySev}

% 6x9
\documentclass{wileySix}


%\usepackage[dvipsone]{graphicx}
\usepackage{verbatim}

\usepackage{threeparttable}
\usepackage{booktabs}
\usepackage{setspace}
\usepackage{subfigure}
\usepackage{graphicx}
\usepackage{slashbox}
\usepackage{multirow}
\usepackage{array}
\usepackage{graphicx}
\usepackage[fleqn]{amsmath}
\usepackage{wrapfig}


\usepackage{color}


% For TimesRoman Math (You must have MathTimes and MathTimes Plus
%                        font sets, order fonts from  www.yandy.com)
% \usepackage[mtbold,noTS1]{m-times}

% For PostScript text
% \usepackage{w-bookps}


%%%%%%%%%%%%%%%%%%%%%%%%%%%%%%
%% Change options here if you want:
%%
%% How many levels of section head would you like numbered?
%% 0= no section numbers, 1= section, 2= subsection, 3= subsubsection
%%==>>
\setcounter{secnumdepth}{8}

%% How many levels of section head would you like to appear in the
%% Table of Contents?
%% 0= chapter titles, 1= section titles, 2= subsection titles,
%% 3= subsubsection titles.
%%==>>
\setcounter{tocdepth}{2}

\setcounter{chapter}{8}

%%%%%%%%%%%%%%%%%%%%%%%%%%%%%%
%
% DRAFT
%
% Uncomment to get double spacing between lines, current date and time
% printed at bottom of page.
\draft

% (If you want to keep tables from becoming double spaced also uncomment
% this):
% \renewcommand{\arraystretch}{0.6}
%%%%%%%%%%%%%%%%%%%%%%%%%%%%%%

%%%%%%% Demo of section head containing sample macro:
%% To get a macro to expand correctly in a section head, with upper and
%% lower case math, put the definition and set the box
%% before \begin{document}, so that when it appears in the
%% table of contents it will also work:

\newcommand{\VT}[1]{\ensuremath{{V_{T#1}}}}

%% use a box to expand the macro before we put it into the section head:

\newbox\sectsavebox
\setbox\sectsavebox=\hbox{\boldmath\VT{xyz}}

%%%%%%%%%%%%%%%%% End Demo


\begin{document}

\offprintinfo{Reinforcement Learning and Approximate Dynamic Programming for Feedback Control,}{Frank L. Lewis and Derong Liu}

\chapter{Nonstationary Stream Data Learning with Imbalanced Class Distribution}
\chapterauthors{Sheng Chen$^\ast$ and Haibo He$^\dag$
\chapteraffil{$^\ast$ Merrill Lynch, Bank of America; $\dag$ University of Rhode Island.}
}
% \footnote{dfdsdsf}

\noindent\textbf{Abstract}

Please include an Abstract for your chapter. Here is just an example...

The ubiquitous imbalanced class distribution occurred in real world data sets has
excited considerable interests in the study of imbalanced learning. However it
is still a relatively uncharted area when it is nonstationary data stream with imbalanced
class distribution that needs to be processed. Difficulties in this case
are generally two-fold. First, dynamically structured learning framework is required
to catch up with the evolution of unstable class concepts, i.e., concept drifts.
Second, imbalanced class distribution over data stream demands a mechanism to
intensify the underrepresented class concepts for improved overall performance.
For instance, in order to design an intelligent spam filtering system, one needs
to not only make the system to be capable of self-tuning its learning parameters
to keep pace with the rapid evolution of spam mail patterns, but also the system
has to tackle the fundamental problem that in some situations normal emails
are severely outnumbered by spam emails, but it is so much more expensive to
misclassify a normal email into spam, e.g., confirmation of a business contract,
than the other way around. This chapter introduces learning algorithms that were
specifically proposed to tackle the problem of learning from nonstationary data set
with imbalanced class distribution. System-level principle and framework of these
methods are described in algorithmic level, the soundness of which are furthered
validated through theoretical analysis as well as simulations on both synthetic and
real-world benchmarks with varied level of imbalanced ratio and noise level.

\section{Introduction}
% no \IEEEPARstart
Learning from data stream has been featured in many practical applications such as network traffic
monitoring and credit fraud identification \cite{Grossberg:1988}...

This is just an example -- please add all your contents here...


\section{sampling approaches}

\subsection{under-sampling}

\section{Acknowledgement}

For any acknowledgement (such as grants, etc.), please add them here...


\bibliographystyle{ieeetr}
% argument is your BibTeX string definitions and bibliography database(s)
\bibliography{Ch9_reference}

\end{document}

